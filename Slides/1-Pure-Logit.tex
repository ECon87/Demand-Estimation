\documentclass[aspectratio=169,t,11pt,table]{beamer}
% xcolor and define colors -------------------------
\usepackage{xcolor}

% https://www.viget.com/articles/color-contrast/
\definecolor{navy}{HTML}{567293}
\definecolor{purple}{HTML}{695693}
\definecolor{ruby}{HTML}{9a2515}
\definecolor{alice}{HTML}{107895}
\definecolor{daisy}{HTML}{EBC944}
\definecolor{coral}{HTML}{F26D21}
\definecolor{kelly}{HTML}{829356}
\definecolor{cranberry}{HTML}{E64173}
\definecolor{jet}{HTML}{131516}
\definecolor{asher}{HTML}{555F61}
\definecolor{slate}{HTML}{314F4F}

% Mixtape Sessions
\definecolor{picton-blue}{HTML}{00b7ff}
\definecolor{violet-red}{HTML}{ff3881}
\definecolor{sun}{HTML}{ffaf18}
\definecolor{electric-violet}{HTML}{871EFF}

\newcommand\pictonBlue[1]{{\color{picton-blue}#1}}
\newcommand\sun[1]{{\color{sun}#1}}
\newcommand\electricViolet[1]{{\color{electric-violet}#1}}
\newcommand\violetRed[1]{{\color{violet-red}#1}}

\newcommand\bgPictonBlue[1]{{\colorbox{picton-blue!20!white}{#1}}}
\newcommand\bgSun[1]{{\colorbox{sun!20!white}{#1}}}
\newcommand\bgElectricViolet[1]{{\colorbox{electric-violet!20!white}{#1}}}
\newcommand\bgVioletRed[1]{{\colorbox{violet-red!20!white}{#1}}}

\def\code#1{\texttt{#1}}

% Zinc from Tailwind Colors
\definecolor{zinc50}{HTML}{fafafa}
\definecolor{zinc100}{HTML}{f4f4f5}
\definecolor{zinc200}{HTML}{e4e4e7}
\definecolor{zinc300}{HTML}{d4d4d8}
\definecolor{zinc400}{HTML}{a1a1aa}
\definecolor{zinc500}{HTML}{71717a}
\definecolor{zinc600}{HTML}{52525b}
\definecolor{zinc700}{HTML}{3f3f46}
\definecolor{zinc800}{HTML}{27272a}
\definecolor{zinc900}{HTML}{18181b}
\definecolor{zinc950}{HTML}{09090b}

% Easily color text
\newcommand\purple[1]{{\color{purple}#1}}
\newcommand\cranberry[1]{{\color{cranberry}#1}}
\newcommand\raspberry[1]{{\color{raspberry}#1}}
\newcommand\orange[1]{{\color{orange}#1}}
\newcommand\navy[1]{{\color{navy}#1}}
\newcommand\teal[1]{{\color{teal}#1}}
\newcommand\kelly[1]{{\color{kelly}#1}}
\newcommand\ruby[1]{{\color{ruby}#1}}
\newcommand\alice[1]{{\color{alice}#1}}
\newcommand\daisy[1]{{\color{daisy}#1}}
\newcommand\coral[1]{{\color{coral}#1}}

% Color background of text
\newcommand\bgNavy[1]{{\colorbox{navy!80!white}{#1}}}
\newcommand\bgOrange[1]{{\colorbox{orange!80!white}{#1}}}
\newcommand\bgTeal[1]{{\colorbox{teal!80!white}{#1}}}
\newcommand\bgPurple[1]{{\colorbox{purple!80!white}{#1}}}
\newcommand\bgKelly[1]{{\colorbox{kelly!80!white}{#1}}}
\newcommand\bgRuby[1]{{\colorbox{ruby!80!white}{#1}}}
\newcommand\bgAlice[1]{{\colorbox{alice!80!white}{#1}}}
\newcommand\bgDaisy[1]{{\colorbox{daisy!80!white}{#1}}}
\newcommand\bgCoral[1]{{\colorbox{coral!80!white}{#1}}}
\newcommand\bgCranberry[1]{{\colorbox{cranberry!80!white}{#1}}}


% Main theme colors
\definecolor{accent}{HTML}{00b7ff}
\definecolor{accent2}{HTML}{871EFF}
\definecolor{gray100}{HTML}{f3f4f6}
\definecolor{gray800}{HTML}{1F292D}


% Beamer Options -------------------------------------

% Background
\setbeamercolor{background canvas}{bg = white}

% Change text margins
\setbeamersize{text margin left = 15pt, text margin right = 15pt} 

% \alert
\setbeamercolor{alerted text}{fg = accent2}

% Frame title
\setbeamercolor{frametitle}{bg = white, fg = zinc900}
\setbeamercolor{framesubtitle}{bg = white, fg = accent}
\setbeamerfont{framesubtitle}{size = \small, shape = \itshape}

% Block
\setbeamercolor{block title}{fg = white, bg = accent2}
\setbeamercolor{block body}{fg = gray800, bg = gray100}

% Title page
\setbeamercolor{title}{fg = zinc600}
\setbeamercolor{subtitle}{fg = accent}

%% Custom \maketitle and \titlepage
\setbeamertemplate{title page}
{
    %\begin{centering}
        \vspace{20mm}
        {\Large \usebeamerfont{title}\usebeamercolor[fg]{title}\inserttitle}\\
        {\large \itshape \usebeamerfont{subtitle}\usebeamercolor[fg]{subtitle}\insertsubtitle}\\ \vspace{10mm}
        {\insertauthor}\\
        {\color{asher}\small{\insertdate}}\\
    %\end{centering}
}

% Table of Contents
\setbeamercolor{section in toc}{fg = accent!70!jet}
\setbeamercolor{subsection in toc}{fg = jet}

% Button 
\setbeamercolor{button}{bg = white, fg = zinc600!80!accent}
\setbeamerfont{button}{family = \ttfamily}
\setbeamercolor{button border}{fg = black}

% Let's you put \beamerbutton in bottom right corner
\usepackage[absolute,overlay]{textpos}
\newcommand\bottomleft[1]{%
  \begin{textblock*}{\paperwidth}(0pt,{\dimexpr\textheight + 5pt})
    \hspace{8pt}\raggedright #1
  \end{textblock*}
}
\newcommand\bottomright[1]{%
  \begin{textblock*}{\paperwidth}(0pt,\textheight)
    \raggedleft #1\hspace{8pt}
  \end{textblock*}
}

% Remove navigation symbols
\setbeamertemplate{navigation symbols}{}

\usepackage{appendixnumberbeamer}
\setbeamercolor{page number in head/foot}{fg=zinc600}
\setbeamertemplate{footline}[frame number]


% Table and Figure captions
\setbeamercolor{caption}{fg=jet!70!white}
\setbeamercolor{caption name}{fg=jet}
\setbeamerfont{caption name}{shape = \itshape}


%% Fix left-margins
\settowidth{\leftmargini}{\usebeamertemplate{itemize item}}
\addtolength{\leftmargini}{\labelsep}

%% enumerate item color
\setbeamercolor{enumerate item}{fg = zinc600}
\setbeamerfont{enumerate item}{size = \small}
\setbeamertemplate{enumerate item}{\insertenumlabel.}

\setbeamercolor{enumerate subitem}{fg = zinc600}
\setbeamercolor{enumerate subsubitem}{fg = zinc600}

%% itemize
\setbeamercolor{itemize item}{fg = zinc600}
\setbeamerfont{itemize item}{size = \small}
\setbeamertemplate{itemize item}[circle]

%% right arrow for subitems
\setbeamercolor{itemize subitem}{fg = zinc600}
\setbeamerfont{itemize subitem}{size = \small}
\setbeamertemplate{itemize subitem}{$\rightarrow$}

\setbeamercolor{itemize subsubitem}{fg = zinc600}
\setbeamerfont{itemize subsubitem}{size = \small}
\setbeamertemplate{itemize subsubitem}[square]




% Links ----------------------------------------------

\usepackage{hyperref}
\hypersetup{
  colorlinks = true,
  linkcolor = accent2,
  filecolor = accent2,
  urlcolor = accent2,
  citecolor = accent2,
}


% Line spacing --------------------------------------
\usepackage{setspace}
\setstretch{1.2}


% \begin{columns} -----------------------------------
\usepackage{multicol}


% Fonts ---------------------------------------------
% Beamer Option to use custom fonts
\usefonttheme{professionalfonts}

% \usepackage[utopia, smallerops, varg]{newtxmath}
% \usepackage{utopia}
\usepackage[sfdefault,light]{roboto}

% Small adjustments to text kerning
\usepackage{microtype}



% Remove annoying over-full box warnings -----------
\vfuzz2pt 
\hfuzz2pt


% Table of Contents with Sections
\setbeamerfont{myTOC}{series=\bfseries, size=\Large}
\AtBeginSection[]{
        \frame{
            \frametitle{Roadmap}
            \tableofcontents[current]   
        }
    }


% Tables -------------------------------------------
% Tables too big
% \begin{adjustbox}{width = 1.2\textwidth, center}
\usepackage{adjustbox}
\usepackage{array}
\usepackage{threeparttable, booktabs, adjustbox}
    
% Fix \input with tables
% \input fails when \\ is at end of external .tex file
\makeatletter
\let\input\@@input
\makeatother

% Tables too narrow
% \begin{tabularx}{\linewidth}{cols}
% col-types: X - center, L - left, R -right
% Relative scale: >{\hsize=.8\hsize}X/L/R
\usepackage{tabularx}
\newcolumntype{L}{>{\raggedright\arraybackslash}X}
\newcolumntype{R}{>{\raggedleft\arraybackslash}X}
\newcolumntype{C}{>{\centering\arraybackslash}X}

% Figures

% \imageframe{img_name} -----------------------------
% from https://github.com/mattjetwell/cousteau
\newcommand{\imageframe}[1]{%
    \begin{frame}[plain]
        \begin{tikzpicture}[remember picture, overlay]
            \node[at = (current page.center), xshift = 0cm] (cover) {%
                \includegraphics[keepaspectratio, width=\paperwidth, height=\paperheight]{#1}
            };
        \end{tikzpicture}
    \end{frame}%
}

% subfigures
\usepackage{subfigure}


% Highlight slide -----------------------------------
% \begin{transitionframe} Text \end{transitionframe}
% from paulgp's beamer tips
\newenvironment{transitionframe}{
    \setbeamercolor{background canvas}{bg=accent!40!black}
    \begin{frame}\color{accent!10!white}\LARGE\centering
}{
    \end{frame}
}


% Table Highlighting --------------------------------
% Create top-left and bottom-right markets in tabular cells with a unique matching id and these commands will outline those cells
\usepackage[beamer,customcolors]{hf-tikz}
\usetikzlibrary{calc}
\usetikzlibrary{fit,shapes.misc}
\usetikzlibrary{decorations.pathreplacing}


% To set the hypothesis highlighting boxes red.
\newcommand\marktopleft[1]{%
    \tikz[overlay,remember picture] 
        \node (marker-#1-a) at (0,1.5ex) {};%
}
\newcommand\markbottomright[1]{%
    \tikz[overlay,remember picture] 
        \node (marker-#1-b) at (0,0) {};%
    \tikz[accent!80!jet, ultra thick, overlay, remember picture, inner sep=4pt]
        \node[draw, rectangle, fit=(marker-#1-a.center) (marker-#1-b.center)] {};%
}



% References ----------------------------------------

%% Bibliography Font, roughly matching aea
\setbeamerfont{bibliography item}{size = \footnotesize}
\setbeamerfont{bibliography entry author}{size = \footnotesize, series = \bfseries}
\setbeamerfont{bibliography entry title}{size = \footnotesize}
\setbeamerfont{bibliography entry location}{size = \footnotesize, shape = \itshape}
\setbeamerfont{bibliography entry note}{size = \footnotesize}

\setbeamercolor{bibliography item}{fg = jet}
\setbeamercolor{bibliography entry author}{fg = accent!60!jet}
\setbeamercolor{bibliography entry title}{fg = jet}
\setbeamercolor{bibliography entry location}{fg = jet}
\setbeamercolor{bibliography entry note}{fg = jet}

%% Remove bibliography symbol in slides
\setbeamertemplate{bibliography item}{}


% Citations ----------------------------------------
\usepackage{natbib}
\bibliographystyle{aer}

\newcommand{\backupbegin}{
  \newcounter{finalframe}
  \setcounter{finalframe}{\value{framenumber}}
}
\newcommand{\backupend}{
  \setcounter{framenumber}{\value{finalframe}}
}

% Math
\usepackage{mathtools}
\usepackage[makeroom]{cancel}
\DeclareMathOperator*{\argmax}{argmax}
\DeclareMathOperator*{\argmin}{argmin}
\renewcommand{\Pr}{\mathbb{P}}
\newcommand{\E}{\mathbb{E}}
\newcommand{\Var}{\mathbb{V}}
\newcommand{\Cov}{\mathbb{C}}
\newcommand{\Corr}{\mathrm{Corr}}
\newenvironment{wideitemize}{\itemize\addtolength{\itemsep}{10pt}}{\enditemize}
\newenvironment{wideitemizeshort}{\itemize}{\enditemize}

\begin{document}

\imageframe{includes/banner.png}

\begin{frame}{Who Am I?}
    \begin{wideitemize}
        \item A fifth-year Economics PhD candidate at Harvard University.
        \pause
        \item Making BLP-style estimation more accessible to researchers.
        \begin{itemize}
            \item Best practices papers \citep{conlon2020best,conlon2023incorporating}.
            \item Open-source Python package\defcitealias{pyblp}{PyBLP}\citepalias{pyblp}.
            \item This course!
        \end{itemize}
    \end{wideitemize}
\end{frame}

\begin{frame}{This Course}
    \begin{wideitemize}
        \item Three days, 6pm-9pm.
        \begin{enumerate}
            \item Today: BLP model, pure logit, price endogeneity.
            \item Wednesday: Mixed logit, identification, numerical best practices.
            \item Friday: Micro BLP, consumer survey data, other extensions.
        \end{enumerate}
        \pause
        \item Please participate and ask questions!
        \begin{itemize}
            \item Ask questions in the Discord chat or raise your hand on Zoom.
        \end{itemize}
        \pause
        \item Three coding exercises, one after each day.
        \begin{itemize}
            \item Try these on your own or with your classmates' help. Use Discord rooms!
            \item I'll do the first two exercises live at the start of days 2 and 3. We'll post solutions.
        \end{itemize}
    \end{wideitemize}
\end{frame}

\begin{frame}{Readings}
    \begin{wideitemize}
        \item There are a lot of possible references for how to do BLP-style estimation.
        \pause
        \item Modern guides:
        \begin{enumerate}
            \item \cite{berry2021foundations}
            \item \cite{conlon2020best}
            \item \cite{conlon2023incorporating}
        \end{enumerate}
        \pause
        \item Foundational guides:
        \begin{enumerate}
            \item \cite*{berry1995automobile}
            \item \cite{nevo2000practitioner}
            \item \cite{petrin2002quantifying}
            \item \cite*{berry2004differentiated}
        \end{enumerate}
        \pause
        \item None of these are required for the course, but I recommend taking a look afterwards.
    \end{wideitemize}
\end{frame}

\begin{frame}{Running Example}
    \begin{wideitemize}
        \item BLP can be used to better understand all sorts of decisions.
        \begin{itemize}
            \item Product purchases, hospital visits, school choice, voting behavior, etc.
        \end{itemize}
        \pause
        \item Typically used for \alert{counterfactual analysis} of something that hasn't happened.
        \begin{itemize}
            \item Need a structural model when we can't just estimate a treatment effect.
        \end{itemize}
        \pause
        \item Running example: \alert{What if we halved an important product's price?}
        \begin{itemize}
            \item Practitioners: Increased sales vs.\ cannibalization?
            \item Regulators: Revenue loss from eliminating a tax?
            \item Academics: Welfare consequences?
        \end{itemize}
    \end{wideitemize}
\end{frame}

\section{The BLP Model}

\begin{frame}{Model Overview}
    \begin{wideitemize}
        \item Model of individuals making a discrete choice from different alternatives.
        \begin{itemize}
            \item Original\defcitealias{berry1995automobile}{BLP (1995)}\citetalias{berry1995automobile} also modeled firm price-setting. We'll focus on demand.
        \end{itemize}
        \pause
        \item Choices are made in \alert{markets} denoted by $t \in \mathcal{T}$.
        \begin{itemize}
            \item Time periods, geographic regions, etc.
        \end{itemize}
        \pause
        \item Each market has \alert{individuals} with types denoted by $i \in \mathcal{I}_t$.
        \begin{itemize}
            \item Different demographics and preferences.
        \end{itemize}
        \pause
        \item Individuals are faced with \alert{choices} denoted by $j \in \mathcal{J}_t$.
        \begin{itemize}
            \item Products, hospitals, candidates, etc.
            \item Outside option $j = 0$: no purchase, no treatment, no vote, etc.
        \end{itemize}
    \end{wideitemize}
\end{frame}

\begin{frame}{Utility Maximization}
    \vspace{-1.25\baselineskip}
    \begin{equation*}
        \max_{j \in \mathcal{J}_t \cup \{0\}} u_{ijt} \only<2->{= \alert<3>{\delta_{jt}} + \alert<4>{\mu_{ijt}} + \alert<5>{\varepsilon_{ijt}}}
    \end{equation*}
    \vspace{-0.5\baselineskip}
    \begin{wideitemize}
        \item Individuals choose an alternative to maximize (indirect) utility $u_{ijt}$.
        \begin{itemize}
            \item We will specify a function for $u_{ijt}$ and use revealed preference to estimate it.
        \end{itemize}
        \pause
        \item Will help to decompose utility into three parts.
        \pause
        \begin{enumerate}
            \item Mean utility $\alert<3>{\delta_{jt}}$: Average preference across all individuals in the market.
            \pause
            \item Systematic heterogeneity $\alert<4>{\mu_{ijt}}$: Different preferences, e.g.\ due to different demographics.
            \pause
            \item Idiosyncratic heterogeneity $\alert<5>{\varepsilon_{ijt}}$: Superimposed noise that accommodates estimation.
        \end{enumerate}
        \pause
        \item We will parameterize $\delta_{jt}$ and $\mu_{ijt}$ and make a convenient assumption about $\varepsilon_{ijt}$.
    \end{wideitemize}
\end{frame}

\begin{frame}{Aggregate Market Shares}
    \vspace{-1.5\baselineskip}
    \begin{minipage}[c][4\baselineskip][c]{\textwidth}
        \begin{equation*}
            \max_{j \in \mathcal{J}_t \cup \{0\}} u_{ijt} = \delta_{jt} + \mu_{ijt} + \varepsilon_{ijt} \only<2->{\quad\implies\quad s_{ijt} = \alt<2>{\Pr_{\varepsilon_{it}}\Big(u_{ijt} \geq u_{ikt} \text{ for all } k \in \mathcal{J}_t \cup \{0\}\Big)}{\frac{\exp(\delta_{jt} + \mu_{ijt})}{\sum_{k \in \mathcal{J}_t \cup \{0\}} \exp(\delta_{kt} + \mu_{ikt})}}}
        \end{equation*}
    \end{minipage}
    \vspace{-0.5\baselineskip}
    \begin{wideitemize}
        \item Assume a convenient distribution for $\varepsilon_{ijt}$: iid type I extreme value.
        \pause
        \begin{itemize}
            \item ``Logit shocks'' are convenient because they give multinomial logit choice probabilities $s_{ijt}$.
        \end{itemize}
        \pause\pause
        \item Want $\mu_{ijt}$ to be sufficiently flexible that this convenient assumption matters little.
        \begin{itemize}
            \item Possible to eliminate $\varepsilon_{ijt}$ but computation gets difficult \citep{berry2007pure}.
        \end{itemize}
        \pause
        \item Each type $i$ is a share $w_{it}$ of the population. Aggregating over them gives market shares.
        \begin{equation*}
            s_{jt} = \sum_{i \in \mathcal{I}_t} w_{it} \cdot s_{ijt}
        \end{equation*}
        \vspace{-1.5\baselineskip}
        \pause
        \item We'll match these to observed quantities $q_{jt} = s_{jt} \cdot M_t$ in our data.
    \end{wideitemize}
\end{frame}

\begin{frame}{Choosing a Market Size}
    \begin{wideitemize}
        \item In our data, we observe quantities $q_{jt} = s_{jt} \cdot M_t$.
        \begin{itemize}
            \item Need to divide by some market size $M_t$ to get our model's market shares $s_{jt}$.
            \item Issue here is that we often don't observe the quantity of outside choices $q_{0t}$.
        \end{itemize}
        \pause
        \item Sometimes the choice of market size is straightforward.
        \begin{itemize}
            \item Market size for drugs to treat a condition is how many people have that condition.
        \end{itemize}
        \pause
        \item But typically, the choice of market size is \alert{neither easy nor innocuous}.
        \begin{itemize}
            \item E.g.\ how many choices of which cereal to buy are made every day in a specific city?
            \item Population $\times$ max cereals per day? Foot traffic estimate $\times$ max cereals per trip?
        \end{itemize}
        \pause
        \item You should try different assumptions and see how they change your results.
        \begin{itemize}
            \item In general, the bigger the market size, the more substitution to the outside good.
            \item We'll learn how to discipline these assumptions with data on day 3.
        \end{itemize}
    \end{wideitemize}
\end{frame}

\begin{frame}{Identification and Normalizations}
    \vspace{-\baselineskip}
    \begin{equation*}
        \max_{j \in \mathcal{J}_t \cup \{0\}} u_{ijt} = \delta_{jt} + \mu_{ijt} + \varepsilon_{ijt}
    \end{equation*}
    \vspace{-0.5\baselineskip}
    \begin{wideitemize}
        \item We will estimate our utility function with \alert{revealed preference}.
        \begin{itemize}
            \item Holding $\mu_{ijt}$ fixed, a higher quantity $q_{jt} > q_{kt}$ implies a higher mean utility $\delta_{jt} > \delta_{kt}$.
        \end{itemize}
        \pause
        \item Utility is invariant to positive affine transformations. Need two normalizations.
        \only<2>{
            \vspace{-0.5\baselineskip}
            \begin{equation*}
                u_{ijt} > u_{ikt} \quad\xLeftrightarrow{b > 0}\quad a + b \cdot u_{ijt} > a + b \cdot u_{ikt}
            \end{equation*}
        }
        \pause
        \begin{enumerate}
            \item[a.] \alert{Level}: We will normalize $u_{i0t} = \varepsilon_{i0t}$, i.e.\ $\delta_{jt} = \mu_{ijt} = 0$ \\ $\implies$ Estimates are relative to outside option utility.
            \pause
            \item[b.] \alert{Scale}: We already normalized $\Var(\varepsilon_{ijt}) = \pi^2 / 6$ when deriving choice probabilities. \\ $\implies$ Estimates are relative to scale of noise.
        \end{enumerate}
        \pause
        \item Now that our model can in theory be identified, how do we estimate it?
    \end{wideitemize}
\end{frame}

\section{Pure Logit Estimation}

\begin{frame}{Pure Logit Model}
    \vspace{-\baselineskip}
    \begin{minipage}[c][4\baselineskip][c]{\textwidth}
        \begin{equation*}
            \max_{j \in \mathcal{J}_t \cup \{0\}} u_{ijt} = \delta_{jt} \only<1>{+ \alert{\cancelto{0}{\mu_{ijt}}}} + \varepsilon_{ijt} \only<2->{\quad\implies\quad \alert<2>{s_{jt}} = \frac{\exp \delta_{jt}}{\alt<2>{\sum_{k \in \mathcal{J}_t \cup \{0\}}}{\alert<3>{1} + \sum_{k \in \mathcal{J}_t}} \exp \delta_{kt}}} \only<4->{\quad\implies\quad \log\frac{s_{jt}}{s_{0t}} = \alert{\delta_{jt}}}
        \end{equation*}
    \end{minipage}
    \vspace{-0.5\baselineskip}
    \begin{wideitemize}
        \item Start with the simplest case: no heterogenous utility. We'll add $\mu_{ijt}$ back on day 2.
        \pause
        \item Market shares simplify. No aggregation over individual types.
        \pause
        \begin{itemize}
            \item The \alert<3>{1} in the denominator is from our level normalization $u_{i0t} = \varepsilon_{i0t}$, i.e.\ $\delta_{jt} = 0$.
        \end{itemize}
        \pause
        \item We can recover mean utilities from observed market shares \citep{berry1994estimating}.
        \begin{itemize}
            \item If we specify a function for $\delta_{jt}$, we'll have a linear regression!
        \end{itemize}
    \end{wideitemize}
\end{frame}

\begin{frame}{Pure Logit Estimating Equation}
    \vspace{-\baselineskip}
    \begin{equation*}
        \log\frac{s_{jt}}{s_{0t}} = \delta_{jt} \only<2->{= \alpha p_{jt} + x_{jt}'\beta + \alert<3>{\xi_{jt}}}
    \end{equation*}
    \vspace{-0.5\baselineskip}
    \begin{wideitemize}
        \item Running example: What if we halved an important product's price?
        \begin{itemize}
            \item In your exercise, products $j$ are breakfast cereals; markets $t$ are city-quarters.
            \item If we estimate the model, we can change $p_{jt}$ and estimate how consumers react.
        \end{itemize}
        \pause
        \item Specify $\delta_{jt}$ as a function of price $p_{jt}$ and other product characteristics $x_{jt}$.
        \begin{itemize}
            \item In your exercise, $p_{jt}$ is per serving; $x_{jt}$ includes a constant, a ``mushy'' dummy, etc.
        \end{itemize}
        \pause
        \item Interpret the regression error $\alert<3>{\xi_{jt}}$ as unobserved product quality not in our data.
        \begin{itemize}
            \item Unobserved characteristics, advertising, average taste variation, ``demand shocks,'' etc.
        \end{itemize}
    \end{wideitemize}
\end{frame}

\begin{frame}{Interpreting Parameters}
    \vspace{-\baselineskip}
    \begin{equation*}
        \log\frac{s_{jt}}{s_{0t}} = \delta_{jt} = \alert<2-3>{\alpha} p_{jt} + x_{jt}'\alert<4>{\beta} + \xi_{jt}
    \end{equation*}
    \vspace{-0.5\baselineskip}
    \begin{wideitemize}
        \item Let's say we estimate this equation. How to interpret our parameter estimates?
        \pause
        \item Prices are in dollars, so the units of $\alpha$ are ``utils'' per dollar. Not very helpful.
        \pause
        \begin{itemize}
            \item Instead, report own-price elasticities, or a quantity-weighted average/median.
            \item You can derive elasticities by differentiating the multinomial logit expression for $s_{jt}$.
        \end{itemize}
        \vspace{\baselineskip}
        \begin{equation*}
            \eta_{jjt} = \frac{\partial\log q_{jt}}{\partial\log p_{jt}} = \frac{\partial q_{jt}}{\partial p_{jt}} \frac{p_{jt}}{q_{jt}} = \frac{\partial s_{jt}}{\partial p_{jt}} \frac{p_{jt}}{s_{jt}} = \alpha \cdot p_{jt} \cdot (1 - s_{jt})
        \end{equation*}
        \vspace{-0.5\baselineskip}
        \pause
        \item If $x_{jt}$ is a ``mushy'' cereal dummy, $\beta$ is ``utils'' from mushyness. Again, not helpful.
        \pause
        \begin{itemize}
            \item Instead, report $\beta / \alpha$, the dollar willingness to pay for mushyness.
        \end{itemize}
    \end{wideitemize}
\end{frame}

\section{Price Endogeneity}

\begin{frame}{Endogeneity Concerns}
    \vspace{-\baselineskip}
    \begin{equation*}
        \delta_{jt} = \alpha p_{jt} + x_{jt}'\beta + \xi_{jt}
    \end{equation*}
    \vspace{-0.5\baselineskip}
    \begin{wideitemize}
        \item In your coding exercise, you'll run an OLS regression of $\delta_{jt}$ on $p_{jt}$ and $x_{jt}$.
        \pause
        \item As usual, if a regressor is correlated with the error, then its coefficient is biased.
        \pause
        \item Typically, we expect price to be strongly correlated with unobserved quality.
        \begin{itemize}
            \item Firms know more than us about demand when setting prices.
            \item Often, $\Cov(p_{jt}, \xi_{jt}) > 0$, so $\hat{\alpha} < 0$ is biased towards zero.
        \end{itemize}
        \pause
        \item Today we'll focus on handling just price endogeneity for simplicity.
    \end{wideitemize}
\end{frame}

\begin{frame}{Fixed Effects}
    \vspace{-\baselineskip}
    \begin{equation*}
        \delta_{jt} = \alpha p_{jt} + x_{jt}'\beta + \xi_{jt}
    \end{equation*}
    \vspace{-0.5\baselineskip}
    \begin{wideitemize}
        \item Adding product and market fixed effects to $x_{jt}$ can help a lot.
        \begin{itemize}
            \item E.g.\ if $p_{jt}$ is correlated with product-specific $\xi_j$ in $\xi_{jt} = \xi_j + \xi_t + \Delta\xi_{jt}$.
            \item But do need multiple observations per product and market to add $\xi_j$ and $\xi_t$.
            \pause
            \item Aside: Related to dynamic panel approach. Let $\xi_{jt} = \rho \xi_{jt-1} + \Delta\xi_{jt}$, estimate $\rho$.
        \end{itemize}
        \pause
        \item Modern grocery scanner datasets have many thousands of products/markets.
        \begin{itemize}
            \item Dummies take too much memory, so we ``absorb'' them, i.e.\ iteratively de-mean.
            \item Stata:\defcitealias{reghdfe}{Reghdfe}\citetalias{reghdfe}. R:\defcitealias{fixest}{Fixest}\citetalias{fixest}.  Python:\defcitealias{pyfixest}{PyFixest}\citetalias{pyfixest}. Coding exercise:\defcitealias{pyblp}{PyBLP}\citetalias{pyblp} via\defcitealias{pyhdfe}{PyHDFE}\citetalias{pyhdfe}.
        \end{itemize}
        \pause
        \item Helpful but insufficient: $\xi_{jt}$ typically varies by product \textit{and} market.
        \begin{itemize}
            \item And if prices don't, then we don't have any variation left to estimate $\alpha$.
        \end{itemize}
    \end{wideitemize}
\end{frame}

\begin{frame}{Instrumental Variables}
    \vspace{-\baselineskip}
    \begin{equation*}
        \delta_{jt} = \alpha p_{jt} + x_{jt}'\beta + \xi_{jt}
    \end{equation*}
    \vspace{-0.5\baselineskip}
    \begin{wideitemize}
        \item With or without fixed effects, a carefully-chosen IV can be a good solution.
        \begin{itemize}
            \item Relevance: $\Cov(p_{jt}, z_{jt}) \neq 0$. Exclusion: $\Cov(\xi_{jt}, z_{jt}) = 0$.
        \end{itemize}
        \pause
        \item Always run a first-stage regression of $p_{jt}$ on $z_{jt}$ and $x_{jt}$.
        \begin{itemize}
            \item Does the sign of the coefficient on $z_{jt}$ make sense?
            \item Is the instrument strong, or should you worry about weak instruments?
        \end{itemize}
        \pause
        \item Many places to look. I'll discuss the most common ones.
    \end{wideitemize}
\end{frame}

\begin{frame}{Typical Instruments for Price}
    \vspace{-\baselineskip}
    \begin{equation*}
        \delta_{jt} = \alpha p_{jt} + x_{jt}'\beta + \xi_{jt}
    \end{equation*}
    \vspace{-0.5\baselineskip}
    \begin{wideitemize}
        \item Typically, prices are marginal costs plus a markup term.
        \only<1>{
            \begin{itemize}
                \item We want valid instruments that shift costs and/or markups.
            \end{itemize}
        }
        \pause
        \item \alert{Cost-shifters}: Measures of input prices, tariffs, etc.
        \only<2>{
            \begin{itemize}
                \item Consumers should only care about them through their effect on prices.
            \end{itemize}
        }
        \pause
        \item \defcitealias{hausman1996valuation}{Hausman}\citetalias{hausman1996valuation}: Current price of the same product averaged across \textit{other} locations.
        \only<3>{
            \begin{itemize}
                \item Need costs to be correlated across locations, but not unobserved quality.
                \item Helpful that retailers tend to do ``uniform pricing'' \citep{dellavigna2019uniform}.
            \end{itemize}
        }
        \pause
        \item \defcitealias{waldfogel2003preference}{Waldfogel}\citetalias{waldfogel2003preference}: Average consumer characteristics in \textit{nearby} locations.
        \only<4>{
            \begin{itemize}
                \item With uniform pricing, your neighbors' demographics will affect your prices.
            \end{itemize}
        }
        \pause
        \item \defcitealias{berry1995automobile}{BLP}\citetalias{berry1995automobile}: Average characteristics $x_{kt}$ of \textit{competing} products $k \neq j$.
        \only<5>{
            \begin{itemize}
                \item Characteristics of competing products affect markups.
                \item We'll come back to these later, since they can also serve a different purpose.
            \end{itemize}
        }
        \pause
        \item I recommend starting with just one. A straightforward cost-shifter if you have it.
    \end{wideitemize}
\end{frame}

\section{Coding Exercise 1}

\begin{frame}{Coding Exercise 1}
    \begin{wideitemize}
        \item Try to do the first exercise before day 2's class, when I'll do it live.
        \begin{enumerate}
            \item Getting set up with Python and PyBLP.
            \item Pure logit estimation.
            \item Running the price cut counterfactual.
        \end{enumerate}
        \pause
        \item When doing the exercise, think critically about the pure logit model's limitations.
        \begin{itemize}
            \item Do the substitution patterns you estimate seem reasonable?
        \end{itemize}
        \pause
        \item If you have time, try the supplemental exercises.
        \begin{itemize}
            \item Statistical inference.
            \item Modeling the supply side.
            \item Checking your code by simulating data.
        \end{itemize}
    \end{wideitemize}
\end{frame}

\backupbegin

\begin{frame}[allowframebreaks,noframenumbering,plain]{References}
    \bibliography{includes/references.bib}
\end{frame}

\backupend

\end{document}
